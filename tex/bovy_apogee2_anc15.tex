\documentclass[12pt,preprint]{aastex}
\usepackage{savesym}
\savesymbol{singlespace}
\savesymbol{doublespace}
\usepackage{setspace}
\usepackage{amssymb,amsmath}
\usepackage{sidecap}
\setlength{\emergencystretch}{2em}%No overflowing references
\newcommand{\ie}{i.e.}
\newcommand{\etal}{et al.}
\newcommand{\dd}{\mathrm{d}}
\newcommand{\eg}{e.g.}
\newcommand{\eqnname}{equation}
\newcommand{\figurenames}{\figurename s}
\newcommand{\sectionname}{$\mathsection$}

\begin{document}

\title{APOGEE-2 Ancillary Proposal: The Far Disk in APOGEE-2 using Low-extinctoin Windows}
\author{P.I.: Jo~Bovy (IAS $\rightarrow$ Toronto);\\
  Institute for Advanced Study, Einstein Drive, Princeton, NJ 08540, USA;\\
  609-734-8369}
\email{bovy@ias.edu}

\section{Co.I.s:}
\begin{itemize}
  \itemsep-.25em
  \begin{spacing}{0.1}
  \item Brett Andrews (PITT PACC, Department of Physics and Astronomy,
University of Pittsburgh, 3941 O’Hara Street, Pittsburgh, PA 15260, USA)
  \item Jonathan C. Bird (Physics and Astronomy Department, Vanderbilt University, 1807 Station B, Nashville, TN 37235, USA)
  \item Diane Feuillet (New Mexico State University, Las Cruces, NM 88003, USA)
  \item Douglas P.~Finkbeiner (Harvard-Smithsonian Center for Astrophysics, 60 Garden Street, Cambridge, MA 02138, USA)
  \item Gregory Green (Harvard-Smithsonian Center for Astrophysics, 60 Garden Street, Cambridge, MA 02138, USA)
  \item Jon Holtzman (New Mexico State University, Las Cruces, NM 88003, USA)
  \item Chao Liu (Key Laboratory of Optical Astronomy, National Astronomical
Observatories, Chinese Academy of Sciences, Datun Road 20A, Beijing 100012, China)
  \item Melissa Ness  (Max-Planck-Institut f\"ur Astronomie, K\"onigstuhl 17, D-69117 Heidelberg, Germany)
  \item David L. Nidever (Department of Astronomy, University of Michigan, Ann Arbor, MI 48109, USA)
  \item Hans-Walter Rix (Max-Planck-Institut f\"ur Astronomie, K\"onigstuhl 17, D-69117 Heidelberg, Germany)
  \item Mathias Schultheis (Laboratoire Lagrange (UMR7293), Universit\'{e} de Nice Sophia Antipolis, CNRS, Observatoire de la C\^{o}te d'Azur, BP 4229, 06304 Nice Cedex 4, France)
\end{spacing}\end{itemize}

\section{External Collaborator Request for Finkbeiner and Green}

BOVY: TO DO

\section{Type of Request}

We are requesting targets in two existing fields as well as a single
new field. Therefore, the type is a combination of 2 and 3.
\newpage

\section{Scientific Justification}

One of APOGEE-2's main scientific objectives is a comprehensive study
of the chemo-dynamical structure of a large volume of the Milky Way's
disk. In particular, the color selection in the main ``Disk''
sub-sample is tuned to produce a larger number of distant stars
(defined as $D\gtrsim$ 6 kpc in the Disk Working Group White Paper) to
extend APOGEE-1's already extensive study of the ``local'' disk ($D
\lesssim$ 6 kpc). However, while the main ``Disk'' target-selection
will produce a large sample of stars at distances between 6 and
$\approx9$ kpc, the expected number of stars at these large distances
in the mid-plane will only be a few dozen, primarily because of the
large extinction.

With the availability of three-dimensional dust maps covering a large
fraction of the sky and a large range of extinctions
\citep[\eg,][]{Marshall06a,Green15a}, it is now possible to identify
low-extinction windows in the inner Milky Way where stars at large
distances can be observed at relatively bright optical and infrared
magnitudes. While the dust is highly filamentary on small scales such
that most of the area of a typical APOGEE pointing in the mid-plane
suffers from high extinction, substantial fractions of a pointing can
cover low extinction regions. We propose here to take advantage of
these windows in already existing APOGEE-2 disk pointings
($l=34^\circ$ and $64^\circ$ as well as in a new pointing centered on
$l=27^\circ$) to reach larger distances along these lines of sight than
would normally be reached. A major motivation of this proposal is that
if this selection is successful, it could be used in a post-Gaia
target selection for APOGEE-2 and future surveys.

A sample of only a few hundred of stars in low-extinction windows
probing distances as far as 16 kpc a few magnitudes below the tip of
the red-giant branch would significantly improve APOGEE-2's
investigation of the large-scale dynamics and metallicity structure of
the disk. Because of the low extinction, such stars would have highly
precise proper motions from Gaia (at large distances proper motions
due to Galactic rotation are a few mas yr$^{-1}$) that combined with
APOGEE’s precise radial velocities allow the study of large-scale
lopsided modes in the disk and therefore more direct constraints on
the axisymmetric rotation (the rotation curve) than will be possible
from Gaia data alone. Similarly, a few hundred stars would allow the
mean metallicity at otherwise inaccessible regions of the Disk to be
mapped leading to much stronger constraints on the azimuthal chemical
homogeneity of the disk. The l=64$^\circ$ field would also sample the
outer disk in a region that is much less affected by the warp than
that at l=180, allowing for a cleaner study of the outer disk in that
region and for stronger constraints on the warp and flaring of the
disk by comparison with the $l=180^\circ$ study.

\section{Feasibility Assessment}

\section{Data Reduction}

\section{Summary of results from previous SDSS ancillary science programs}

None.

\newpage

\section{Target Information}

\begin{thebibliography}{}
\bibitem[Green \etal(2015)]{Green15a}
  Green,~G., Schlafly,~E., Finkbeiner,~D., \etal\ 2015, \apj, submitted BOVY: UPDATE
\bibitem[Marshall \etal(2006)]{Marshall06a}
  Marshall, D.~J., Robin, A.~C., Reyl{\'e}, C., Schultheis, M., \& Picaud, S.\ 2006, \aap, 453, 635
\end{thebibliography}

\end{document}
